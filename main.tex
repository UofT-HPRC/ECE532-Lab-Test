\documentclass[11pt]{article}
\usepackage{geometry}
\geometry{letterpaper, margin=1in}
\usepackage[utf8]{inputenc}
\usepackage{graphicx}
\usepackage{upgreek}
\usepackage{url}
\usepackage[T1]{fontenc}
\usepackage{amsmath}
\usepackage[colorinlistoftodos]{todonotes}
\usepackage[
backend=biber,
style=ieee,
sorting=none
]{biblatex}
 
\addbibresource{assign.bib}

\title{ECE532S Digital Systems Design \\ \vspace{0.4cm}
       \Large Lab Test 1 Description - Individual Project \\ \vspace{0.4cm}
       \small Last Updated: July, 2019}
\author{ }
\date{ }

\begin{document}
\maketitle
\vspace{-1cm}

The purpose of this test is to show that you have mastered some of the basics of using Vivado. In advance of
the test you should prepare the following two tasks. Although you may work on these tasks as a group, each
person in the group should be prepared to demonstrate and answer questions about any part of any of these
tasks. Individual grades will be assigned based on your familiarity with the tasks and on three questions that
each of you will be asked. The questions will come from what you did and learned in the tutorials, from the
tasks described below and basic Verilog (HDL) programming.

This is worth 10\% of your final grade. If you do not have this exercise prepared in advance, you will be
assigned a grade of 0\%. Please ensure that you have signed up for a demo time slot. Contact your TA if you
do not have one yet or do not remember your time.



\section{Microblaze System}
Build a MicroBlaze system that writes a 32-bit value to the base address of an AXI BRAM memory
and then reads the value back for display in hexadecimal on the eight 7-segment displays of the
Nexys4 DDR board. You will need to consult the Nexys4 DDR manual to learn how the displays
work. When constructing the block design, constrain the pins manually without the board files as you
did in the first MicroBlaze tutorial. You may use a GPIO to connect to the displays.

\section{Simple Module and Simulation}
Build a module using the HDL of your choice that inputs a stream of 4-bit numbers and outputs the
largest of the numbers currently input. Assume all numbers are positive integers.
The block has the following ports:
\\

\textbf{clk (input)} the clock

\textbf{reset\_b (input)} an active low reset signal

\textbf{InputValid (input)} value at data input is valid

\textbf{DataInput[3:0] (input)} data input

\textbf{Largest[3:0] (output)} the largest of the input values so far

\noindent
The block operates as follows:

\begin{enumerate}
    \item The \textbf{reset\_b} signal is set to 0 for a few cycles to reset the state ma
    \noindentchine.
    \item On a positive clock edge, if \textbf{InputValid} is true, then \textbf{DataInput[3:0]} has the next valid input.
    \item The output \textbf{Largest[3:0]} is 0 after reset and then shows the largest of the numbers that have been input so far. This output changes one cycle after the corresponding number has been input.
    \item The circuit continues to operate until a reset.
\end{enumerate}

\noindent
Your task is to:

\begin{enumerate}
    \item Simulate this block using the Vivado simulator, or Modelsim. Note that Modelsim will require some extra configuration.
    \item Demonstrate the use of a simple test bench using your HDL that exercises your module.
    \item Develop enough tests to convince someone that your circuit is working.
    \item Show the results of some tests with a waveform viewer.
    \item Use Vivado to synthesize the block and show that it can synthesize without errors. You do not need to actually load a chip and show this working on the hardware.
\end{enumerate}









% EXTRA: Use this to build polynomial system
% EXTRA: Consider timing
\newpage
\printbibliography

\end{document}
